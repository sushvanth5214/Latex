\documentclass[12pt,-letter paper]{article}
\usepackage{siunitx}
\usepackage{setspace}
\usepackage{gensymb}
\usepackage{xcolor}
\usepackage{caption}
%\usepackage{subcaption}
\doublespacing
\singlespacing
\usepackage[none]{hyphenat}
\usepackage{amssymb}
\usepackage{relsize}
\usepackage[cmex10]{amsmath}
\usepackage{mathtools}
\usepackage{amsmath}
\usepackage{commath}
\usepackage{amsthm}
\interdisplaylinepenalty=2500
%\savesymbol{iint}
\usepackage{txfonts}
%\restoresymbol{TXF}{iint}
\usepackage{wasysym}
\usepackage{amsthm}
\usepackage{mathrsfs}
\usepackage{txfonts}
\let\vec\mathbf{}
\usepackage{stfloats}
\usepackage{float}
\usepackage{cite}
\usepackage{cases}
\usepackage{subfig}
%\usepackage{xtab}
\usepackage{longtable}
\usepackage{multirow}
%\usepackage{algorithm}
\usepackage{amssymb}
%\usepackage{algpseudocode}
\usepackage{enumitem}
\usepackage{mathtools}
%\usepackage{eenrc}
%\usepackage[framemethod=tikz]{mdframed}
\usepackage{listings}
%\usepackage{listings}
\usepackage[latin1]{inputenc}
%%\usepackage{color}{   
%%\usepackage{lscape}
\usepackage{textcomp}
\usepackage{titling}
\usepackage{hyperref}
%\usepackage{fulbigskip}   
\usepackage{tikz}
\usepackage{graphicx}
\lstset{
  frame=single,
  breaklines=true
}
\let\vec\mathbf{}
\usepackage{enumitem}
\usepackage{graphicx}
\usepackage{siunitx}
\let\vec\mathbf{}
\usepackage{enumitem}
\usepackage{graphicx}
\usepackage{enumitem}
\usepackage{tfrupee}
\usepackage{amsmath}
\usepackage{amssymb}
\usepackage{mwe} % for blindtext and example-image-a in example
\usepackage{wrapfig}
\graphicspath{{figs/}}
\providecommand{\mydet}[1]{\ensuremath{\begin{vmatrix}#1\end{\vmatrix}}}
\providecommand{\myvec}[1]{\ensuremath{\begin{bmatrix}#1\end{\bmatrix}}}
\providecommand{\cbrak}[1]{\ensuremath{\left\{#1\right\}}}
\providecommand{\brak}[1]{\ensuremath{\left(#1\right)}}
\providecommand{\pr}[1]{\ensuremath{\Pr\left(#1\right)}}
\providecommand{\qfunc}[1]{\ensuremath{Q\left(#1\right)}}
\providecommand{\sbrak}[1]{\ensuremath{{}\left[#1\right]}}
\providecommand{\lsbrak}[1]{\ensuremath{{}\left[#1\right]}}
\providecommand{\rsbrak}[1]{\ensuremath{{}\left.#1\right}}
\providecommand{\brak}[1]{\ensuremath{\left(#1\right)}}
\providecommand{\lbrak}[1]{\ensuremath{\left(#1\right.}}
\providecommand{\rbrak}[1]{\ensuremath{\left.#1\right)}}
\providecommand{\cbrak}[1]{\ensuremath{\left\{#1\right\}}}
\providecommand{\lcbrak}[1]{\ensuremath{\left\{#1\right.}}
\providecommand{\rcbrak}[1]{\ensuremath{\left.#1\right\}}}
\begin{document}
\begin{center}
\section*{Algebra}
\end{center}

\begin{enumerate}


\item If the sum of zeroes of the polynomial $ p\brak x = 2x^2 - k\sqrt2x+1 $ is${\sqrt2} $,then value of k is:
\begin{enumerate}
    \item $ \sqrt{2} $
    \item $2$
    \item $ 2  \sqrt{2} $
    \item $ \frac{1}{2} $
 \end{enumerate}

\item If the roots of the equation $ax^2 + bx + c = 0$,$a \neq 0$ are real and equal, then which of the following relations is true?
\begin{enumerate}    
    \item $a = \frac{b^2}{c}$
    \item $b^2 = ac$                                                                                    \item $ac = \frac{b^2}{4}$
    \item $c = \frac{b^2}{a}$
\end{enumerate}

\item In an A.P., if the first term $a = 7$, $n$th term $a_{n} = 84$, and the sum of the first $n$ terms $s_{n} = \frac{2093}{2}$, then $n$ is equal to:
\begin{enumerate}
    \item $22$
    \item $24$
    \item $23$
    \item $26$
\end{enumerate}

\item The zeroes of a polynomial $x^2 + px + q$ are twice the zeroes of the polynomial $4x^2 - 5x - 6$. The value of $p$ is:
	\begin{enumerate}   
\item $-\frac{5}{2}$
    \item $\frac{5}{2}$
    \item $-5$
    \item $10$
	\end{enumerate}
\newpage
 \item In the given figure, graphs of two linear equations are shown. The pair of these linear equations is:
\begin{figure}[!ht]
\centering
\includegraphics[width=\columnwidth]{Image1.jpg}
\caption{}
\label{fig:enter-label}
\end{figure}
\begin{enumerate}
    \item consistent with a unique solution.
    \item consistent with infinitely many solutions.
    \item inconsistent.
    \item inconsistent but can be made consistent.
\end{enumerate}
\
\begin{center}
\section*{Statistics and Probability}  
\end{center}
\item If the probability of a player winning a game is 0.79, then the probability of his losing the same game is:
\begin{enumerate}    
    \item $1.79$
    \item $0.31$
    \item $0.21	$                                                           
    \item $0.21$
\end{enumerate}

\item From the data $1, 4, 7, 9, 16, 21, 25$, if all the even numbers are removed, then the probability of getting at random a prime number from the remaining is:
	\begin{enumerate}
\item $\frac{2}{5}$
    \item $\frac{1}{5}$
    \item $\frac{1}{7}$
    \item $\frac{2}{7}$
	\end{enumerate}

\item For some data $x_{1}, x_{2}, \dots, x_{n}$ with respective frequencies $f_{1}, f_{2}, \dots, f_{n}$, the value of $\sum_{i}^{n}f_{i} \brak{x_{i} - \overline{x}}$ is equal to:
	\begin{enumerate}    
\item $n \overline{x}$
    \item $1$
    \item $\Sigma f_{i}$
    \item $0$
	\end{enumerate}

\item The middle-most observation of every data arranged in order is called:
	\begin{enumerate}    
\item mode
    \item median
    \item mean
    \item deviation
\end{enumerate}
\newpage
\item Two dice are rolled together. The probability of getting a sum of numbers on the two dice as $2$, $3$, or $5$ is:
	\begin{enumerate}    
\item $\frac{7}{36}$
    \item $\frac{11}{36}$
    \item $\frac{5}{36}$
    \item $\frac{4}{9}$
	\end{enumerate}

\begin{center}
\section*{Geometry}  
\end{center}
\item A solid sphere is cut into two hemispheres. The ratio of the surface areas of the sphere to that of the two hemispheres taken together is:
	\begin{enumerate}    
\item $1:1$
    \item $1:4$
    \item $2:3$
    \item $3:2$
	\end{enumerate}

\item The volume of the largest right circular cone that can be carved out from a solid cube of edge $2 \, \text{cm}$ is:
	\begin{enumerate}    
\item $\frac{4\pi}{3} \, \mathrm{cu cm}$
    \item $\frac{5\pi}{3} \, \mathrm{cu cm}$
    \item $\frac{8\pi}{3} \, \mathrm{cu cm}$
    \item $\frac{2\pi}{3} \, \mathrm{cu cm}$
	\end{enumerate}

\item \textbf{Assertion (A):} The tangents drawn at the end points of a diameter of a circle are parallel.

\textbf{Reason (R):} The diameter of a circle is the longest chord.
\begin{enumerate}
    \item Both Assertion (A) and Reason (R) are true, and Reason (R) is the correct explanation of Assertion (A).
    \item Both Assertion (A) and Reason (R) are true, but Reason (R) is not the correct explanation for Assertion (A).
    \item Assertion (A) is true, but Reason (R) is false.
    \item Assertion (A) is false, but Reason (R) is true.
\end{enumerate}
\begin{center}
\section*{ Co-ordinate Geometry}    
\end{center}
\item $AD$ is a median of $\triangle ABC$ with vertices $A(5, -6)$, $B(6, 4)$, and $C(0, 0)$. The length of $AD$ is equal to:
	\begin{enumerate}    
\item $\sqrt{68}$ units
    \item $2\sqrt{15}$ units
    \item $\sqrt{101}$ units
    \item $10$ units
	\end{enumerate}

 \item If the distance between the points $\brak{3, -5}$ and $\brak{x, -5}$ is 15 units, then the values of $x$ are:
    \begin{enumerate}
    \item $12, -18$
    \item $-12, 18$
    \item $18, 5$
    \item $-9, -12$
    \end{enumerate}

    \item The center of a circle is at $\brak{2, 0}$. If one end of a diameter is at $\brak{6, 0}$, then the other end is at:
	\begin{enumerate}    
		\item $\brak{0, 0}$
		\item $\brak{4, 0}$
		\item $\brak{-2, 0}$
		\item $\brak{-6, 0}$
	\end{enumerate}
\begin{center}
\section*{Number System}
\end{center}
\item If two positive integers $p$ and $q$ can be expressed as $p = 18a^2 b^4$ and $q = 20a^3 b^2$ where $a$ and $b$ are prime numbers, then $\text{LCM}(p, q)$ is:
	\begin{enumerate}    
\item $2a^2 b^2$
    \item $180a^2 b^2$
    \item $12a^2  b^2$
    \item $180a^3 b^4$
	\end{enumerate}
 
\begin{center}
\section*{Trigonomentry}
\end{center}
\item If $\sec\theta - \tan\theta = m$, then the value of $\sec\theta + \tan\theta$ is:
\begin{enumerate}
    \item ${1}-\frac{1}{m}$                                                       
    \item $m^2 - 1$
    \item $\frac{1}{m}$
    \item $-m$
\end{enumerate}

 \item If $\cos(\alpha + \beta) = 0$ then the value of $\cos\left(\frac{\alpha + \beta}{2}\right)$ is equal to:
 \begin{enumerate}
    \item $\frac{1}{\sqrt{2}}$
    \item $\frac{1}{2}$
    \item $0$
    \item $\sqrt{2}$
\end{enumerate}
\end{enumerate}
\end{document}
