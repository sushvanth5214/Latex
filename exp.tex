\documentclass[12pt,-letter paper]{article}
\usepackage{siunitx}
\usepackage{setspace}
\usepackage{gensymb}
\usepackage{xcolor}
\usepackage{caption}
%\usepackage{subcaption}
\doublespacing
\singlespacing
\usepackage[none]{hyphenat}
\usepackage{amssymb}
\usepackage{relsize}
\usepackage[cmex10]{amsmath}
\usepackage{mathtools}
\usepackage{amsmath}
\usepackage{commath}
\usepackage{amsthm}
\interdisplaylinepenalty=2500
%\savesymbol{iint}
\usepackage{txfonts}
%\restoresymbol{TXF}{iint}
\usepackage{wasysym}
\usepackage{amsthm}
\usepackage{mathrsfs}
\usepackage{txfonts}
\let\vec\mathbf{}
\usepackage{stfloats}
\usepackage{float}
\usepackage{cite}
\usepackage{cases}
\usepackage{subfig}
%\usepackage{xtab}
\usepackage{longtable}
\usepackage{multirow}
%\usepackage{algorithm}
\usepackage{amssymb}
%\usepackage{algpseudocode}
\usepackage{enumitem}
\usepackage{mathtools}
%\usepackage{eenrc}
%\usepackage[framemethod=tikz]{mdframed}
\usepackage{listings}
%\usepackage{listings}
\usepackage[latin1]{inputenc}
%%\usepackage{color}{   
%%\usepackage{lscape}
\usepackage{textcomp}
\usepackage{titling}
\usepackage{hyperref}
%\usepackage{fulbigskip}   
\usepackage{tikz}
\usepackage{graphicx}
\lstset{
  frame=single,
  breaklines=true
}
\let\vec\mathbf{}
\usepackage{enumitem}
\usepackage{graphicx}
\usepackage{siunitx}
\let\vec\mathbf{}
\usepackage{enumitem}
\usepackage{graphicx}
\usepackage{enumitem}
\usepackage{tfrupee}
\usepackage{amsmath}
\usepackage{amssymb}
\usepackage{mwe} % for blindtext and example-image-a in example
\usepackage{wrapfig}
\graphicspath{{figs/}}
\providecommand{\mydet}[1]{\ensuremath{\begin{vmatrix}#1\end{\vmatrix}}}
\providecommand{\myvec}[1]{\ensuremath{\begin{bmatrix}#1\end{\bmatrix}}}
\providecommand{\cbrak}[1]{\ensuremath{\left\{#1\right\}}}
\providecommand{\brak}[1]{\ensuremath{\left(#1\right)}}
\providecommand{\pr}[1]{\ensuremath{\Pr\left(#1\right)}}
\providecommand{\qfunc}[1]{\ensuremath{Q\left(#1\right)}}
\providecommand{\sbrak}[1]{\ensuremath{{}\left[#1\right]}}
\providecommand{\lsbrak}[1]{\ensuremath{{}\left[#1\right]}}
\providecommand{\rsbrak}[1]{\ensuremath{{}\left.#1\right}}
\providecommand{\brak}[1]{\ensuremath{\left(#1\right)}}
\providecommand{\lbrak}[1]{\ensuremath{\left(#1\right.}}
\providecommand{\rbrak}[1]{\ensuremath{\left.#1\right)}}
\providecommand{\cbrak}[1]{\ensuremath{\left\{#1\right\}}}
\providecommand{\lcbrak}[1]{\ensuremath{\left\{#1\right.}}
\providecommand{\rcbrak}[1]{\ensuremath{\left.#1\right\}}}
\begin{document}

\begin{center}
        \section*{SECETION-A}
\end{center}
\begin{enumerate}

\item If the sum of zeroes of the polynomial $ p\brak x = 2x^2 - k\sqrt2x+1 $ is ${\sqrt2} $,then value of k is:
    \item[(a)] $ \sqrt{2} $
    \item[(b)] $2$
    \item[(c)] $ 2  \sqrt{2} $
    \item[(d)] $ \dfrac{1}{2} $

\item If the probability of a player winning a game is 0.79, then the probability of his losing the same game is:
    \item[(a)] $1.79$
    \item[(b)] $0.31$
    \item[(c)] $0.21\% $                                                                                                                                      \item[(d)] $0.21$

\item If the roots of the equation $ax^2 + bx + c = 0$,$a \neq 0$ are real and equal, then which of the following relations is true?
    \item[(a)] $a = \dfrac{b^2}{c}$
    \item[(b)] $b^2 = ac$                                                                                                                                     \item[(c)] $ac = \dfrac{b^2}{4}$
    \item[(d)] $c = \dfrac{b^2}{a}$

\item In an A.P., if the first term $a = 7$, $n$th term $a_{n} = 84$, and the sum of the first $n$ terms $s_{n} = \frac{2093}{2}$, then $n$ is equal to:
    \item[(a)] $22$
    \item[(b)] $24$
    \item[(c)] $23$
    \item[(d)] $26$

\item If two positive integers $p$ and $q$ can be expressed as $p = 18a^2 b^4$ and $q = 20a^3 b^2$ where $a$ and $b$ are prime numbers, then $\text{LCM}(p, q)$ is:
    \item[(a)] $2a^2 b^2$
    \item[(b)] $180a^2 b^2$
    \item[(c)] $12a^2  b^2$
    \item[(d)] $180a^3 b^4$

\item $AD$ is a median of $\triangle ABC$ with vertices $A(5, -6)$, $B(6, 4)$, and $C(0, 0)$. The length of $AD$ is equal to:
    \item[(a)] $\sqrt{68}$ units
    \item[(b)] $2\sqrt{15}$ units
    \item[(c)] $\sqrt{101}$ units
    \item[(d)] $10$ units

\item If $\sec\theta - \tan\theta = m$, then the value of $\sec\theta + \tan\theta$ is:
\item[(a)] ${1}-\frac{1}{m}$                                                                                                                                  \item[(b)] $m^2 - 1$
    \item[(c)] $\frac{1}{m}$
    \item[(d)] $-m$

\item From the data $1, 4, 7, 9, 16, 21, 25$, if all the even numbers are removed, then the probability of getting at random a prime number from the remaining is:
    \item[(a)] $\frac{2}{5}$
    \item[(b)] $\frac{1}{5}$
    \item[(c)] $\frac{1}{7}$
    \item[(d)] $\frac{2}{7}$

\item For some data $x_{1}, x_{2}, \dots, x_{n}$ with respective frequencies $f_{1}, f_{2}, \dots, f_{n}$, the value of $\sum_{i}^{n}\brak{ f_{i} x_{i} - \overline{x}}$ is equal to:
    \item[(a)] $n \overline{x}$
    \item[(b)] $1$
    \item[(c)] $\Sigma f_{i}$
    \item[(d)] $0$

\item The zeroes of a polynomial $x^2 + px + q$ are twice the zeroes of the polynomial $4x^2 - 5x - 6$. The value of $p$ is:
    \item[(a)] $-\frac{5}{2}$
    \item[(b)] $\frac{5}{2}$
    \item[(c)] $-5$
    \item[(d)] $10$

\item If the distance between the points $(3, -5)$ and $(x, -5)$ is 15 units, then the values of $x$ are:
    \item[(a)] $12, -18$
    \item[(b)] $-12, 18$
    \item[(c)] $18, 5$
    \item[(d)] $-9, -12$

\item If $\cos(\alpha + \beta) = 0$ then the value of $\cos\left(\frac{\alpha + \beta}{2}\right)$ is equal to:
    \item[(a)] $\frac{1}{\sqrt{2}}$
    \item[(b)] $\frac{1}{2}$
    \item[(c)] $0$
    \item[(d)] $\sqrt{2}$

\item A solid sphere is cut into two hemispheres. The ratio of the surface areas of the sphere to that of the two hemispheres taken together is:
    \item[(a)] $1:1$
    \item[(b)] $1:4$
    \item[(c)] $2:3$
    \item[(d)] $3:2$

\item The middle-most observation of every data arranged in order is called:
    \item[(a)] mode
    \item[(b)] median
    \item[(c)] mean
    \item[(d)] deviation

\item The volume of the largest right circular cone that can be carved out from a solid cube of edge $2 \, \text{cm}$ is:
    \item[(a)] $\frac{4\pi}{3} \, \text{cu cm}$
    \item[(b)] $\frac{5\pi}{3} \, \text{cu cm}$
    \item[(c)] $\frac{8\pi}{3} \, \text{cu cm}$
    \item[(d)] $\frac{2\pi}{3} \, \text{cu cm}$

\item Two dice are rolled together. The probability of getting a sum of numbers on the two dice as $2$, $3$, or $5$ is:
    \item[(a)] $\frac{7}{36}$
    \item[(b)] $\frac{11}{36}$
    \item[(c)] $\frac{5}{36}$
    \item[(d)] $\frac{4}{9}$

\item The center of a circle is at $(2, 0)$. If one end of a diameter is at $(6, 0)$, then the other end is at:
    \item[(a)] $(0, 0)$
    \item[(b)] $(4, 0)$
    \item[(c)] $(-2, 0)$
    \item[(d)] $(-6, 0)$

\item In the given figure, graphs of two linear equations are shown. The pair of these linear equations is:
\begin{figure}[!ht]
\centering
\includegraphics[width=0.67 \textwidth]{Image1.jpg}
\label{fig:enter-label}
\end{figure}

    \item[(a)] consistent with a unique solution.
    \item[(b)] consistent with infinitely many solutions.
    \item[(c)] inconsistent.
    \item[(d)] inconsistent but can be made consistent.


\textbf{Directions:}

\textbf{In Q. No. 19 , a statement of Assertion (A) is followed by a statement of Reason (R). Choose the correct option.}

    \item[(a)] Both Assertion (A) and Reason (R) are true, and Reason (R) is the correct explanation of Assertion (A).
    \item[(b)] Both Assertion (A) and Reason (R) are true, but Reason (R) is not the correct explanation for Assertion (A).
    \item[(c)] Assertion (A) is true, but Reason (R) is false.
    \item[(d)] Assertion (A) is false, but Reason (R) is true.
\item \textbf{Assertion (A):} The tangents drawn at the end points of a diameter of a circle are parallel.

\textbf{Reason (R):} The diameter of a circle is the longest chord.


\end{enumerate}
\end{document}                                                               
